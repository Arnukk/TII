
 \vspace{-5pt}
\section{Conclusion}\label{sec:concl}

This paper presented efficient algorithms for event demand-based response in microgrid with provable approximation guarantee. 
A two-stage method was suggested that combines the advantages of two different approaches: the one-dimensional projection approach (1DPA) which guarantees a good approximation ratio but is computationally more demanding; and the greedy ratio approach (GRA) which is computationally very efficient, but has  worse approximation guarantee. This paper compares these methods with the methods currently used in practice, such as greedy utility, or greedy demand. The simulation results show the superiority of the suggested methods under various practical settings. The proposed approach can be applied to microgrid with a large number of customers.

%While GRA is worse than the 1DPA in terms of approximation guarantee, it has the advantage that it can be modified to work directly with the voltage and frequency fluctuations, i.e., without requiring an explicit value for the capacity. Investigating this further will be the subject of future work.  