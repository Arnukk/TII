\section{Introduction}

\iffalse
Demand response has become one of the key enabling technologies for smart grids. With the increasing demand response incentives set by utilities, more customers are subscribing to the various demand response schemes. In {\em event-driven demand response programs}, participating customers agree to allow load shedding in response to requests by the grid operators. Load shedding can be determined in multiple ways, for instance, based on (1) the bids (representing the valuations of retaining the power demand) from customers, or (2) the compensation (stipulated as a penalty from the operators) paid to customers. 
Demand response has become one of the key enabling technologies for smart grids. With the increasing demand response incentives set by utilities, more customers are subscribing to the various demand response schemes. In {\em event-driven demand response programs}, participating customers agree to allow load shedding in response to requests by the grid operators. Load shedding can be determined in multiple ways, for instance, based on (1) the bids (representing the valuations of retaining the power demand) from customers, or (2) the compensation (stipulated as a penalty from the operators) paid to customers. 

However, with growing customer participation, the problems of determining the solutions of minimum operator cost and maximum customer valuation become computationally complex (even with hundreds of customers). This paper proposes efficient algorithms for event driven demand response management for AC electric islanded systems (e.g., microgrids). In these systems, it is important to optimally shed loads as fast as possible to maintain microgrid stability and minimize costs, considering a combination of active and reactive power. An efficient two-stage algorithm is proposed where the optimal loads to be shed during islanded operation is determined. The first stage relies on a greedy approach that is capable of determining a close-to-optimal load shedding scheme rapidly to maintain microgrid stability. The second stage relies on a one-dimensional projection algorithm that can further improve the optimality solution of the first stage, when more response time is permitted. The algorithms are corroborated extensively by simulations with up to thousands of customers.

Load reduction requests are sent out when electricity demand is high enough to put grid reliability at risk, or rising demand requires the imminent activation of expensive/unreliable generation assets. Participating customers are compensated based on their flexibility and load. 
\fi


Distributed Generation  (DG) is one of the key enabling technologies for smart grids. As the number of installed DGs increase in the system, microgrid implementation becomes an attractive and valuable option. Microgrids typically are medium-to-low voltage networks with integrated DG, capable of operating in grid connected or islanded mode.  Designing a smart grid with the capability of operating in an islanded mode can enhance system reliability and power quality. Nevertheless, there is a high probability that a microgrid once initiated will be short of power, consequently resulting in significant voltage and frequency deviations, and leading to microgrid instability. 

Demand Response (DR) programs can be broadly classified into three classes: economic demand response, emergency demand response, and ancillary services demand response. Emergency demand response \cite{Ref6} is utilized when there is insufficient supply of power to meet the available demand, especially for microgrids. Demand response is a key feature for smart grids and can be used to alter loads during contingency conditions.  Demand response has proven to have many benefits including decrease in price variations \cite{Ref1}, increased reliability \cite{Ref2}, congestion management \cite{Ref3} and security enhancement \cite{Ref4}. In \cite{Ref5}, an {\em event-based demand response} algorithm has been used to improve microgrid operation lifetime by modifying the load consumptions. The method relies on an optimization model that selects the best combination of remedial actions, including load curtailment and load transfer to neighboring substation in order to reduce loading during contingencies. In \cite{Ref7}, a discrete event based simulation framework is developed to examine whether the capacity of the existing power system can meet the demand of plug-in hybrid vehicles. The power system limited generation and transmission capacities are considered to be the major constraints \cite{Ref7}. In \cite{Ref8}, an emergency demand response model was developed to maximize DR benefits while satisfying the reserved capacity constrains for an interconnected power system. In \cite{Ref9}, simultaneous implementation of the unit commitment algorithm and emergency demand response has been considered and tested on an interconnected power system. 


Sudden islanding of microgrids can cause high imbalances between the local generation and demand and thus, management strategies are needed to ensure the microgrid endurance during its autonomous operation \cite{Ref10}. Innovative demand response strategies for microgrids can contribute to improve the microgrid stability especially during emergency conditions \cite{Ref10}. The emergency demand response method, proposed in \cite{Ref10,Ref12}, is based on local frequency measurement to switch on/off a group of loads. However, such methods do not take into account customer utility and operator costs. 


Prior studies, focusing on microgrid, only considered systems with small number of loads, and thus optimizing the operation of the microgrid within a short time frame during emergency conditions (within milliseconds) is possible. However, with growing customer participation, the problem of determining the solutions of optimal load shedding for customers becomes computationally complex (even with hundreds of customers). This paper proposes a two-stage event-based demand response algorithm for microgrids with a large number of customers. In order to assure microgrid stable operation as a result of sudden demand imbalance, the first stage utilizes a greedy algorithm, capable of obtaining a close to optimal solution, to determine the load to be curtailed in microseconds base. This is designed considering (1) the bids (representing the valuations of retaining the power demand) from customers, or (2) the compensation (stipulated as a penalty from the operators) paid to customers. The second stage relies on a one-dimensional projection algorithm that can further improve the optimality solution of the first stage, when more response time is permitted. The proposed event-based approach is tested by simulations for microgrids with up to thousands of customers.



This paper is structured as follows.
Section~\ref{sec:model} provides the model definitions and notations needed. Then efficient algorithms are presented in Section~\ref{sec:algs}. Section~\ref{sec:sims} evaluates the algorithms by simulations with a large number of customers. Finally, the conclusion is provided in Section~\ref{sec:concl}.
